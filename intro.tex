%\section*{}
\label{sec:introduction}
\dropcap{A}s data become cheaper to gather and store, researchers have become increasingly reliant on computational workflows requiring skills in statistical modeling, machine learning and scalable computation. In addition, the recent reproducibility crises in several scientific fields has motivated the acquisition of skills in open science and the design of reproducible workflows \cite[e.g.][]{pashler2012,baker2016}.
Formal university curricula have been relatively slow to offer courses in these important topics: the slack in this area has often been picked-up by extra-curricular, ad-hoc efforts such as workshops \cite{demasi2017}.
Well-known examples are the Software and Data Carpentry workshops providing training in research computing skills through a volunteer instructor program \cite{b:wilson-swc-lessons-2016,teal2015data}.
At the same time, there has been a rise in the number of statistical and computational courses designed for specific scientific disciplines.
Examples include the \textit{Summer School in Statistics for Astronomers}\footnote{\url{http://astrostatistics.psu.edu/su16/}}, the Google Earth Engine User Summits\footnote{\url{https://events.withgoogle.com/google-earth-engine-user-summit-2017/}}, as well as a variety of project-focused (rather than pedagogical) meetings, such as the dotAstronomy meetings\footnote{\url{http://dotastronomy.com}}.
Shorter, but similar-spirit meetings have been held in conjunction with conferences, such as the Hack Days at the annual American Astronomical Society meetings, the Brainhack hackathons associated with the Organization for Human Brain Mapping and the Society for Neuroscience\cite{Cameron_Craddock2016-wc}, and a hackathon at the American Geophysical Union meeting\footnote{\url{http://onlinelibrary.wiley.com/doi/10.1002/2014EO480004/pdf}}. 
In general, many of these events tend to emphasize either a more traditional pedagogical class and lecture methodology, or emphasize a focus on ad-hoc projects that participants work on during the event (Figure \ref{fig:hackspectrum}).
Pedagogically-focused events follow a classic academic model where novices learn new skills from experts. This model tends to focus on a one-way flow of information from instructor to student, and are usually targeted towards participants in the training phase of their career. On the other end of the spectrum, project-focused workshops emphasize collaborative activities using existing skills, leading to the common perception that they are designed for technical experts, and this may limit their audience. 
To bridge this gap, we describe here a model that we have implemented: ``Hack Weeks'' that aim to capitalize on the advantages of each of these models.
These week-long events combine structured periods focused on pedagogy (often with an emphasis on statistical and computational techniques) with less structured periods devoted to hacks and creative projects, with the goal of encouraging collaboration and learning among people at various stages of their career. 
Our hackweek model shares some similarities to those summer schools (e.g., Advanced Course on Computational Neuroscience, Okinawa Computational Neuroscience Course, Woods Hole Computational Neuroscience Summer course) that combine individual or group project work with teaching methods that transfer knowedge from experts to early-career researchers.
However, a distinguishing factor of our hackweeks in that they tend to be less structured and more participant-driven, so that the learning content is co-created between organizers and participants based on the unique experience of people in the room. 
This looser structure is designed to facilitate a multi-directional transfer of knowledge between a diverse group of people in different stages of their career and with different backgrounds (see also \ref{sec:participants}). The flexibility of our approach results in shifts along the continuum of pedagogy and open project work across hackweek disciplines, and from year to year within a specific discipline (Figure \ref{fig:hackspectrum}).

\begin{figure}
\begin{center}
\includegraphics[width=8cm]{NewHackSpectrum.pdf}
\caption{{\bf Comparison of Extracurricular Workshop Models}: Different types of events lie on a spectrum between the emphasis on pedagogy (e.g. Software Carpentry workshop) to an emphasis on project-based/hack-based activities (e.g. at science-oriented hackathons). Hack weeks also vary in the degree of emphasis on projects (Astro hack week, which included unstructured time for group projects from the first day) or pedagogy (Neuro hack week, which included an entire first day of data science tutorials, with less time for projects)}
\label{fig:hackspectrum}
\end{center}
\end{figure}

As of the publication of this paper, we have run eight such hack week events: four focused on Astronomy, two focused on Neuroscience, and two focused on Geoscience.
Here we share the philosophy behind the hack week model, results from surveys of participants, practical lessons we have learned in organizing these events, and recommendations for future hack weeks in other disciplines. Supplementary materials (SM) provide additional details on the practical aspects of organizing these events.
