%\section*{}
\label{sec:introduction}
\dropcap{A}s data become cheaper to gather and store, researchers have become increasingly reliant on computational workflows requiring skills in statistical modeling, machine learning and scalable computation. In addition, recent concerns about reproducibility crises motivate the acquisition of skills in open science and the design of reproducible workflows \cite[e.g.][]{pashler2012,baker2016}.
Formal university curricula have been relatively slow to offer courses in these important topics and this vacuum is often filled by extra-curricular, ad-hoc, less formal workshops \cite{demasi2017}.
Well-known examples include Software and Data Carpentry workshops, that provide training in research computing through a volunteer instructor program \cite{b:wilson-swc-lessons-2016,teal2015data}.
Meanwhile, there is an increase in statistical and computational courses designed for specific scientific disciplines, such as the \textit{Summer School in Statistics for Astronomers}\footnote{\url{http://astrostatistics.psu.edu/su16/}}, the Google Earth Engine User Summits\footnote{\url{https://events.withgoogle.com/google-earth-engine-user-summit-2017/}}, as well as a variety of project-focused (rather than pedagogical) meetings, such as the dotAstronomy meetings\footnote{\url{http://dotastronomy.com}}.
Shorter meetings are also held in conjunction with conferences, such as the Hack Days at the annual American Astronomical Society meetings, the Brainhack hackathons associated with the meetings of the Organization for Human Brain Mapping and the Society for Neuroscience\cite{Cameron_Craddock2016-wc}, and a hackathon at the American Geophysical Union meeting\footnote{\url{http://onlinelibrary.wiley.com/doi/10.1002/2014EO480004/pdf}}.
In general, many of these events tend to emphasize either more traditional pedagogical class and lecture methodologies, or focus on ad-hoc projects developed during the event (Figure \ref{fig:hackspectrum}).
Pedagogically-focused events follow a classic academic model where novices learn new skills from experts. This model tends to focus on a one-way flow of information from instructor to student, and are usually targeted towards participants in the training phase of their career. On the other end of the spectrum, project-focused workshops emphasize collaborative activities using existing skills, leading to the common perception that they are designed for technical experts. This may limit their audience.
To bridge this gap, we describe here a model that we have implemented: ``Hack Weeks'' that aim to capitalize on the advantages of each of these models.
These week-long events combine structured periods focused on pedagogy (often with an emphasis on statistical and computational techniques) and less structured periods devoted to hacks and creative projects, with the goal of encouraging collaboration and learning among people at various stages of their career.
\renewcommand{\floatpagefraction}{.3}%
\renewcommand{\dblfloatpagefraction}{.3}%

\begin{figure}[h!]
\begin{center}
\includegraphics[width=8cm]{NewHackSpectrum.pdf}
\caption{Different types of events lie on a spectrum between an emphasis on pedagogy (e.g. Software Carpentry workshop) and an emphasis on project-based/hack-based activities (e.g. at science-oriented hackathons). Hack weeks also vary in the degree of emphasis on projects (e.g. Astro Hack Week) or pedagogy (e.g. Neuro Hack Week).}
\label{fig:hackspectrum}
\end{center}
\end{figure}

We have run eight such hack week events: four focused on Astronomy, and two each focused on Neuroscience and Geoscience.
Here we share the philosophy behind the hack week model, results from surveys of participants, practical lessons we have learned in organizing these events, and recommendations for future hack weeks. Supplementary materials (SM) provide additional details on the practical aspects of organizing these events.
