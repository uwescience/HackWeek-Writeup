\section*{Introduction}
\label{sec:introduction}

As data becomes cheaper to gather and store, research across a wide range of disciplines has become increasingly reliant on computational workflows involving a familiarity with aspects of statistical modeling, machine learning, scalable computation, and related skills.
Formal university curricula have been relatively slow to offer courses in these important topics: the slack in this area has often been picked-up by extra-curricular, third-party workshops.
Well-known examples are the Software add Data Carpentry workshops providing training in research computing skills through a volunteer instructor program  \cite{b:wilson-swc-lessons-2016,teal2015data}.
At the same time, there has been a rise in the number of domain-specific courses focusing on statistics and computation within their field.
Examples include the \textit{Summer School in Statistics for Astronomers}\footnote{\url{http://astrostatistics.psu.edu/su16/}}, the Google Earth Engine User Summits\footnote{\url{https://events.withgoogle.com/google-earth-engine-user-summit-2017/}}, and more project-focused than pedagogical meetings, such as the dotAstronomy meetings\footnote{\url{http://dotastronomy.com}}.
Shorter, but similar-spirit meetings have started in conjunction with conferences, such Hack Days at the annual American Astronomical Society meetings, the Brainhack hackathons that take place in conjunction with meetings of the Organization for Human Brain Mapping and the Society for Neuroscience\cite{Cameron_Craddock2016-wc}, and a hackathon at the American Geophysical Union meeting\footnote{\url{http://onlinelibrary.wiley.com/doi/10.1002/2014EO480004/pdf}}.
Generally, pedagogically-focused events follow a classic academic model where novices learn new skills from experts, while project-focused workshops emphasize collaborative activities using existing skills.
A disadvantage of the pedagogical model is that it can tends to focus on a one-way flow of information from instructor to student, and can discount the potential contributions by students.
A disadvantage of the project model is the common perception that the week is designed for technical experts, which may discourage others from attending.
In 2014, we started Astro Hack Week to try to fill the gaps between these models.
The hack week model combines pedagogy (often focused on statistical and computational techniques) with room for hacks and creative projects, with the goal of encouraging collaboration and learning among people at various stages of their career.

\begin{figure}
\begin{center}
\includegraphics[width=9cm]{fig/HackSpectrum}
\caption{Comparison of Extracurricular Workshop Models}
\label{fig:hackspectrum}
\end{center}
\end{figure}

As of the publication of this paper, we have run eight such hack week events: four focused on Astronomy, two focused on Neuroscience, and two focused on Geoscience.
Below we will share some of the philosophy behind the hack week model, practical lessons we have learned in organizing these events, and recommendations for future hack weeks in other disciplines.
