%\section*{}
\label{sec:introduction}
As data become cheaper to gather and store, researchers have become increasingly reliant on computational workflows requiring skills in statistical modeling, machine learning and scalable computation. In addition, the recent reproducibility crises in several scientific fields has motivated the acquisition of skills in open science and the design of reproducible workflows \cite[e.g.][]{pashler2012,frye2015,gezelter2015,baker2016}.
Formal university curricula have been relatively slow to offer courses in these important topics: the slack in this area has often been picked-up by extra-curricular, ad-hoc efforts such as workshops \cite{demasi2017}.
Well-known examples are the Software and Data Carpentry workshops providing training in research computing skills through a volunteer instructor program \cite{b:wilson-swc-lessons-2016,teal2015data}.
At the same time, there has been a rise in the number of statistical and computational courses designed for specific scientific disciplines.
Examples include the \textit{Summer School in Statistics for Astronomers}\footnote{\url{http://astrostatistics.psu.edu/su16/}}, the Google Earth Engine User Summits\footnote{\url{https://events.withgoogle.com/google-earth-engine-user-summit-2017/}}, as well as a variety of project-focused (rather than pedagogical) meetings, such as the dotAstronomy meetings\footnote{\url{http://dotastronomy.com}}.
Shorter, but similar-spirit meetings have been held in conjunction with conferences, such as the Hack Days at the annual American Astronomical Society meetings, the Brainhack hackathons associated with the Organization for Human Brain Mapping and the Society for Neuroscience\cite{Cameron_Craddock2016-wc}, and a hackathon at the American Geophysical Union meeting\footnote{\url{http://onlinelibrary.wiley.com/doi/10.1002/2014EO480004/pdf}}.
In general, events to date have been designed withing either a pedagogical or project-based framework.
Pedagogically-focused events follow a classic academic model where novices learn new skills from experts, while project-focused workshops emphasize collaborative activities using existing skills.
A disadvantage of the pedagogical model is that it tends to focus on a one-way flow of information from instructor to student, and can discount the potential contributions by students.
A disadvantage of the project model is the common perception that the week is designed for technical experts, which may discourage others from attending.
In 2014, we initiated an alternative model of ``Hack Weeks'' that aim to combine the best of each of these models.
These week-long events combine pedagogy (often focused on statistical and computational techniques) with hacks and creative projects, with the goal of encouraging collaboration and learning among people at various stages of their career.

\begin{figure}
\begin{center}
\includegraphics[width=9cm]{fig/HackSpectrum}
\caption{Comparison of Extracurricular Workshop Models}
\label{fig:hackspectrum}
\end{center}
\end{figure}

As of the publication of this paper, we have run eight such hack week events: four focused on Astronomy, two focused on Neuroscience, and two focused on Geoscience.
Here we share the philosophy behind the hack week model, results from surveys of participants, practical lessons we have learned in organizing these events, and recommendations for future hack weeks in other disciplines (supplementary materials provide additional details on the practical aspects of organizing these events).
