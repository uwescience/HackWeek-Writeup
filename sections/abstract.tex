\begin{abstract}
Across almost all scientific disciplines, the instruments that record our experimental data and the methods required for storage and data analysis are rapidly increasing in complexity.
This gives rise to the need for scientific communities to adapt on shorter time scales than traditional university curricula allow for, and therefore requires new modes of knowledge transfer.
At the same time, as disciplines become more specialized, data analysis problems often tend to be fairly universal.
It is thus desirable to foster exchange of ideas and computational workflows across disciplines.
Among the different approaches put forward toward addressing these issues, hack weeks have in recent years emerged as both an effective tool for training researchers in modern data analysis workflows, and in fostering exchange between subdisciplines within and between scientific fields.
While interpreted differently in the various disciplines where they have been implemented, all events consist of a common core of three components: tutorials in state-of-the-art methodology, peer-learning and free-form project work.
In this paper, we present the concept of a hack week in the larger context of scientific meetings and point out similarities and differences to traditional conferences.
We motivate the need for such an event and present in detail its strengths and challenges.
We find that hack weeks are successful at cultivating collaboration and the exchange of knowledge.
Participants self-report that these events help them both in their day-to-day research as well as their careers.
Based on our results, we conclude that hack weeks present an effective, easy-to-implement, fairly low-cost tool to positively impact data analysis literacy in academic disciplines, foster collaboration and cultivate best practices.
\end{abstract}
