\documentclass[9pt,twocolumn,twoside,lineno]{pnas-new}
% Use the lineno option to display guide line numbers if required.
% Note that the use of elements such as single-column equations
% may affect the guide line number alignment.

\templatetype{pnasresearcharticle} % Choose template
% {pnasresearcharticle} = Template for a two-column research article
% {pnasmathematics} = Template for a one-column mathematics article
% {pnasinvited} = Template for a PNAS invited submission

\usepackage{subcaption}
\usepackage{caption}
\usepackage{tabularx}
\usepackage{multicol}
\usepackage{microtype}
\usepackage{booktabs}
\usepackage{threeparttable}
\usepackage{rotating}

\title{Hack Weeks as a model for Data Science Education and Collaboration}

% Use letters for affiliations, numbers to show equal authorship (if applicable) and to indicate the corresponding author
\author[a,b,c,e,1]{Daniela Huppenkothen}
\author[d,e]{Anthony Arendt}
\author[c,b,f,g]{David W. Hogg}
\author[h]{Karthik Ram}
\author[e]{Jake VanderPlas}
\author[e]{Ariel Rokem}

\affil[a]{DIRAC Institute, Department of Astronomy, University of Washington, 3910 15th Ave NE, Seattle, WA 98195, USA}
\affil[b]{Center for Data Science, New York University, 65 5th Avenue, 7th Floor, New York, NY 10003, USA}
\affil[c]{Center for Cosmology and Particle Physics, New York University, 726 Broadway, 10th Floor, New York, NY 10003, USA}
\affil[d]{Polar Science Center/Applied Physics Laboratory, University of Washington, 1013 NE 40th Street, Box 355640, Seattle, WA 98105-6698}
\affil[e]{The University of Washington eScience Institute, The WRF Data Science Studio, Physics/Astronomy Tower, 6th Floor, 3910 15th Ave NE, Campus Box 351570, University of Washington, Seattle, WA 98105, USA}
\affil[f]{Max-Planck-Institut f\"ur Astronomie, K\"onigstuhl 17, D-69117 Heidelberg}
\affil[g]{Center for Computational Astrophysics, Flatiron Institute, 162 5th Ave, New York, NY 10010, USA}
\affil[h]{Berkeley Institute for Data Science \& Berkeley Initiative in Global Change Biology, University of California, Berkeley,  Berkeley CA 94720}

% Please give the surname of the lead author for the running footer
\leadauthor{Huppenkothen}

% Please add here a significance statement to explain the relevance of your work
\significancestatement{As scientific disciplines grapple with more data sets of rapidly increasing complexity and size, new approaches are urgently required to introduce new statistical and computational tools into research communities and improve the cross-disciplinary exchange of ideas. In this paper, we introduce a new type of scientific workshop, called a \textit{hack week}, which allows for fast dissemination of new methodologies into scientific communities, and fosters exchange and collaboration within and between disciplines. We present implementations of this concept in astronomy, neuroscience and geoscience, and show that hack weeks produce positive learning outcomes, foster lasting collaborations, yield scientific results and promote positive attitudes towards open science.}

% Please include corresponding author, author contribution and author declaration information
\authorcontributions{The paper was conceptualized by Jake VanderPlas, David W. Hogg and Karthik Ram. Daniela Huppenkothen, Ariel Rokem and Anthony Arendt performed the studies and the data analysis. All authors contributed to the final manuscript.}

\authordeclaration{The authors declare no conflicts of interest.}
%\equalauthors{\textsuperscript{1}A.O.(Author One) and A.T. (Author Two) contributed equally to this work (remove if not applicable).}
\correspondingauthor{\textsuperscript{1}To whom correspondence should be addressed. E-mail: dhuppenk@uw.edu}

% Keywords are not mandatory, but authors are strongly encouraged to provide them. If provided, please include two to five keywords, separated by the pipe symbol, e.g:
\keywords{Data science $|$ Education $|$ Reproducibility $|$ Interdisciplinary Collaboration $|$ Astronomy $|$ Neuroscience $|$ Geosciences}

%\begin{abstract}
%Please provide an abstract of no more than 250 words in a single paragraph. Abstracts should explain to the general reader the major contributions of the article. References in the abstract must be cited in full within the abstract itself and cited in the text.
%\end{abstract}
\begin{abstract}
Across many scientific disciplines, methods for recording, storing and analyzing data are rapidly increasing in complexity.
Facility in data science tools to manage this complexity requires training in new programming languages and frameworks, as well as immersion in new modes of interaction that foster data sharing and collaborative software development and exchange across disciplines.
Learning these skills from traditional university curricula can be challenging because most courses are not designed to evolve on time scales that can keep pace with rapidly shifting data science methods.
Here we present the concept of a hack week as a novel and effective model offering opportunitites for networking and community building, education in state-of-the-art data science methods and immersion in collaborative project work.
We find that hack weeks are successful at cultivating collaboration and facilitating the exchange of knowledge.
Participants self-report that these events help them both in their day-to-day research as well as their careers.
Based on our results, we conclude that hack weeks present an effective, easy-to-implement, fairly low-cost tool to positively impact data analysis literacy in academic disciplines, foster collaboration and cultivate best practices.
\end{abstract}



\dates{This manuscript was compiled on \today}
\doi{\url{www.pnas.org/cgi/doi/10.1073/pnas.XXXXXXXXXX}}

\begin{document}


% Optional adjustment to line up main text (after abstract) of first page with line numbers, when using both lineno and twocolumn options.
% You should only change this length when you've finalised the article contents.
\verticaladjustment{-2pt}

\maketitle
\thispagestyle{firststyle}
\ifthenelse{\boolean{shortarticle}}{\ifthenelse{\boolean{singlecolumn}}{\abscontentformatted}{\abscontent}}{}

%\section*{}
\label{sec:introduction}
\dropcap{A}s data become cheaper to gather and store, researchers have become increasingly reliant on computational workflows requiring skills in statistical modeling, machine learning and scalable computation. In addition, recent concerns about reproducibility crises motivate the acquisition of skills in open science and the design of reproducible workflows \cite[e.g.][]{pashler2012,baker2016}.
Formal university curricula have been relatively slow to offer courses in these important topics and this vacuum is often filled by extra-curricular, ad-hoc, less formal workshops.
Well-known examples include Software and Data Carpentry workshops, that provide training in research computing through a volunteer instructor program \cite{b:wilson-swc-lessons-2016,teal2015data}.
Meanwhile, there is an increase in statistical and computational courses designed for specific scientific disciplines, such as the \textit{Summer School in Statistics for Astronomers}\footnote{\url{http://astrostatistics.psu.edu/su16/}}, the Google Earth Engine User Summits\footnote{\url{https://events.withgoogle.com/google-earth-engine-user-summit-2017/}}, as well as a variety of project-focused (rather than pedagogical) meetings, such as the dotAstronomy meetings\footnote{\url{http://dotastronomy.com}}.
Shorter meetings are also held in conjunction with conferences, such as the Hack Days at the annual American Astronomical Society meetings, the Brainhack hackathons associated with the meetings of the Organization for Human Brain Mapping and the Society for Neuroscience\cite{Cameron_Craddock2016-wc}, and a hackathon at the American Geophysical Union meeting\footnote{\url{http://onlinelibrary.wiley.com/doi/10.1002/2014EO480004/pdf}}.
In general, many of these events tend to emphasize either more traditional pedagogical class and lecture methodologies, or focus on ad-hoc projects developed during the event (Figure \ref{fig:hackspectrum}).
Pedagogically-focused events follow a classic academic model where novices learn new skills from experts. This model tends to focus on a one-way flow of information from instructor to student, and are usually targeted towards participants in the training phase of their career. On the other end of the spectrum, project-focused workshops emphasize collaborative activities using existing skills, leading to the common perception that they are designed for technical experts. This may limit their audience.
To bridge this gap, we describe here a model that we have implemented: ``Hack Weeks'' that aim to capitalize on the advantages of each of these models.
These week-long events combine structured periods focused on pedagogy (often with an emphasis on statistical and computational techniques) and less structured periods devoted to hacks and creative projects, with the goal of encouraging collaboration and learning among people at various stages of their career.
\renewcommand{\floatpagefraction}{.3}%
\renewcommand{\dblfloatpagefraction}{.3}%

\begin{figure}[h!]
\begin{center}
\includegraphics[width=8cm]{NewHackSpectrum.pdf}
\caption{Different types of events lie on a spectrum between an emphasis on pedagogy (e.g. Software Carpentry workshop) and an emphasis on project-based/hack-based activities (e.g. at science-oriented hackathons). Hack weeks also vary in the degree of emphasis on projects (e.g. Astro Hack Week) or pedagogy (e.g. Neuro Hack Week).}
\label{fig:hackspectrum}
\end{center}
\end{figure}

We have run eight such hack week events: four focused on Astronomy, and two each focused on Neuroscience and Geoscience.
Here we share the philosophy behind the hack week model, results from surveys of participants, practical lessons we have learned in organizing these events, and recommendations for future hack weeks. Supplementary materials (SM) provide additional details on the practical aspects of organizing these events.

\section*{What is a hack week?}

%Our hack weeks combine structured, tutorial-style instruction with informal education and peer learning opportunities occurring within projects and hacks. Additionally, they provide opportunities to build new collaborations and network.
Our hack weeks combine structured, tutorial-style instruction with open-ended project work, providing opportunities for peer learning, networking, and for building collaborations.
In a space spanned by pedagogical focus as one dimension and focus on project work as the other, the hack weeks we have organized are designed to lie somewhere in between traditional summer schools and hackathons, where we believe they fill a space not currently fully addressed by existing models (Figure \ref{fig:hackspectrum}).

The hackathon, a time-bounded, collaborative event that brings together participants around a shared challenge or learning objective \cite{Decker2015}, forms one primary axis of our events.
Hackathons originated from the open-source software movement and have historically focused on software and technology development.
In recent years hackathons have evolved into a model providing opportunities for intensive, interdisciplinary collaboration \cite{Groen2015-cj} and education \cite{Kienzler2015-zu,Lamers2014-xf} in the sciences.
Core elements of hackathons include opportunities for networking, strengthening social ties, and building community connections, both within and across disciplines.
Building on these core elements, hackathons have been implemented in different ways depending on the overall purpose, mode of participation, style of work environment and participant motivation \cite{Drouhard2017}.
%``Catalytic'' hackathons seek novel project ideas aimed at solving a tractable, well-defined challenge.
%``Contributive'' hackathons seek to improve to an existing effort through focused work on discrete tasks, for example to make up for deficiencies in an ongoing project.
%Finally, ``Communal'' hackathons place a strong focus on building a culture of practice and developing resources within an existing community, often defined by a specific domain of knowledge.

Summer schools have been designed to excel in transfer of knowledge from experts in the field to (early-career) researchers: they often serve as an entry point for scientists who aim to expand their research into a new area or switch fields. They are excellent at giving participants a reasonably deep understanding of a topic or field in a short amount of time. Within this concept, learning can take many forms, including traditional lecture formats but also hands-on project work, often in teams (e.g., Advanced Course on Computational Neuroscience, Okinawa Computational Neuroscience Course, Woods Hole Computational Neuroscience Summer course).

Our hack weeks extend the scientifically-focused, communal hackathon model into a space that includes a strong element of pedagogy and peer learning. They aim to synthesize different goals and strategies from both models: they are more participant-driven than a summer school, but have a stronger focus on pedagogy than a hackathon. Where a summer school is often organized around a framework of lectures and tutorials known in advance, hack weeks leave the majority of time to be designed by participants, under careful facilitation of the organizers. Tutorials at hack weeks often serve as an entry point into a topic for further exploration and learning. %This is closely related to the nature of the material taught at these events. Data science is a new, emerging area combining knowledge and skills from many different fields, only a fraction of which can be communicated within a short tutorial. As organizers, we make a curated selection of a small number of tutorials each year, and encourage participants to use the unstructured time to self-organize tutorials on any topic as needed.%, often on topics vastly different to those discussed in the formal lectures.

Hack weeks uniquely allow organizers to tailor the content of the workshop to the needs of the participants in an ad-hoc fashion, including the number and content of tutorials. This way, the group as a whole can respond quickly or react to unforeseen challenges and opportunities. They encourage participants to self-organize in many different forms: experts working with other experts, mentoring relationships between experts and non-experts, or study groups among non-experts, to name but a few. Hack weeks also allow participants to experiment with projects and ideas beyond their day-to-day research: For example, our hack weeks explicitly encourage projects around outreach and work aimed at improving the scientific community itself.

There is, however, a major risk in the lack of focus: by wanting to do many things at once, a hack week might potentially not do any of them well. Because tutorials are not necessarily the major focus of a hack week, the knowledge gained by participants in these tutorials may be shallow. A hack week carries a much larger risk of failure if objectives and expectations are not set by organizers well in advance, and clearly communicated to participants, because they often require significant preparation from the side of participants.
Because of these risks, organizers face a much larger degree of uncertainty, and need to be prepared to focus much of their energy on thoughtful selection and management of participants, and facilitation of the large range of different types of activities at any given time (see also the SM, Section 4.2.4).

We note that the terminology for these events is constantly evolving, and that the ``hackathon'' concept may have implicit connotations that are disfavored in some communities. We also note that all of these events live on a constantly changing continuum, depending on the requirements of the scientific domain within which they live. For example, Neuro Hack Week is moving toward a more traditional summer school model, while Astro Hack Week has strengthened its focus on projects and hacks in recent iterations.

\section*{Why run a Hack Week?}

\noindent\textbf{Education and Training}: While some hack weeks are more focused more on education than others (see Figure \ref{fig:hackspectrum}), skill-development in the form of tutorials as well as informal and peer learning is often a component. Furthermore, lateral knowledge transfer \cite{b:wilson-swc-lessons-2016} through collaboration provides an opportunity to learn skills that are not described in papers and software implementations.

\noindent\textbf{Tool Development}: Hack weeks present an opportunity for scientific software developers to meaningfully engage with users and critically evaluate applications to particular scientific issues.

\noindent\textbf{Community Building}: Hack weeks are an opportunity to catalyze community development through a shared interest in solving computational challenges with open source software. They allow computationally-minded researchers to break from the isolation of their institutions and spark new collaborations.

\noindent\textbf{Interdisciplinary research}: Intensive, time-bounded collaborative events are an opportunity to experiment with concepts, questions, and methods that span boundaries within and across disciplines. Despite the fact that interdisciplinary experiments are impactful \cite{Hall2012-hi}, they are often discouraged in traditional academic environments \cite{Sung2003-go}.

\noindent\textbf{Recruitment and Networking}: Hack weeks are a melting pot of participants from academia, government, and industry and provide numerous opportunities for networking. Close collaboration in diverse groups exposes skills that might be suitable for careers outside of a narrow domain.

\noindent\textbf{It's fun}: Hack weeks provide a respite from routine and a low-stress venue to learn new skills and attempt high-risk projects.

\noindent Note that the reasons for participants to attend a hack week are as diverse as the reasons for running such an event.
Beginner participants may attend primarily to learn a new technique, while others may attend to gain experience in mentoring, or to focus on an existing project already in progress (for more details on setting objectives, see SM, Section 4.1.2).

\section*{Audience and Participant Selection}
Hack weeks differ from traditional conferences or summer schools in that knowledge transfer occurs across many levels of seniority, disciplinary boundaries, and novelty of the topics discussed.
In addition, a substantial amount of hack week content is generated during the event itself, requiring active participation from participants.
Therefore in order to maximize learning outcomes and collaborative exchanges, it is crucial that the participant selection process be carried out with considerable care.

In our experience, a participant group that is diverse across categories of diversity, gender, discipline and career track helps to ensure we meet these objectives.
To achieve this diversity, we advocate for a selection process that is as quantitative and transparent as possible, enabling participants to hold organizers accountable for their selection decisions.
%Transparency is necessary for applicants to understand acceptance/rejection decisions, and accountability is of crucial importance for the detection of inherent biases in the selection, which may harm both the event's success as well as the larger community.

Research particularly in the hiring literature has shown that cohort selection is most effective and unbiased when selection procedures are as quantitative as possible~\cite{sunstein2015wiser,bohnet2016}.
In practice, there are different approaches to counteract intrinsic human biases and provide transparency.
Because human decision makers tend to be swayed by unrelated characteristics including name~\cite{bertrand2004} or gender~\cite{mossracusin2012}, an initial merit selection blinded to demographic characteristics can be an effective way to counteract certain biases. A merit selection could then perform via scores given independently by members of the organizing committee based on a set of pre-defined, explicit selection criteria.
Since human decision makers also routinely overestimate their ability to forecast future performance of a candidate, ~\cite{highhouse2008} it may be beneficial to subsequently set a fairly tolerant threshold for acceptance, and select the cohort via e.g.\ via an algorithm, imposing outside constraints on the selection based on the goals of the workshop.

One solution to the latter problem is implemented in the software \textit{entrofy}\footnote{\url{http://github.com/dhuppenkothen/entrofy}}).
The algorithm aims to find a group of participants that together match as closely as possible a requested distribution on specified dimensions (e.g., career stage, geographic location, etc.), to meet pre-set fractions set by the organizers.
%For example, organizers may require that half of the participants (or as close as possible to that) be graduate students, while also maximizing the number of different countries from which participants originate.
It is worth noting that this algorithm is vulnerable toward biases in two ways: firstly, humans will set the target fractions for any category of interest.
Any human biases involved in setting these target fractions will be perpetuated in the selection procedure.
Secondly, perhaps more obviously, the algorithm can only act on information that has been collected.
Biased participant sets may still result from selection procedures that fail to include crucial categories. %For example, it would be difficult to produce a student-heavy participant set for a summer school if the algorithm has no information about academic seniority, and impossible to correct gender bias in the pool of applicants, if no information is available about the gender of participants.

Blinding in the initial merit selection step is most effective at counteracting biases when the hack week targets a very broad population.
For more targeted workshops, organizers should be mindful that blinding to demographics might be counterproductive~\cite{behaghel2015unintended} if it excludes participants who might have had less exposure to certain technologies or fewer opportunities to learn certain skills.
In this case, it may be beneficial to include demographic characteristics in the first stage.
Because systemic biases likely also enter at the application stage (where underrepresented groups may be less likely to apply) organizers should consider oversampling traditionally disenfranchised groups compared to the population of applicants.

No matter the selection procedures used, we encourage organizers to critically examine their cohort selection, experiment with new approaches, and routinely evaluate the different steps in their procedure%.  For example, merit scores can be interrogated for shared biases, and one can plot the demographics both before and after the initial selection step to diagnose whether certain groups are preferentially selected or excluded
(see also the Supplementary materials for more details on how the previous hack weeks approached this problem).
%One way to maximize transparency in the selection process is to minimize human decision making steps that introduce biases, and to . 
%We work to achieve this by first assessing the merit of each candidate with respect to the overall goals of the hack week.
%We try to minimize bias in this step by blinding ourselves to a candidate's other attributes, including name and other personal information, and assess their candidacy based soley on questions asked specifically for this purpose.
%When doing this procedure for a large enough sample, it is unlikely that the resulting pool of acceptable candidates is smaller than the number of available spaces at the workshop.

%The second step in the selection procedure then requires tie-breaking between equally acceptable candidates.
%It is here where one may impose outside constraints on the selection based on the goals of the workshop.
%If multiple competing constraints are considered, this task essentially becomes a complex optimization problem, for which algorithms exist that will outperform any human selection procedure.

%One solution to this optimization procedure is implemented in the software \textit{entrofy}\footnote{\url{http://github.com/dhuppenkothen/entrofy}}.

\section*{Themes}

To date, all hack weeks we have organized have been subject-specific, i.e.\ aimed at bringing together a community with a shared scientific interest, such as neuroscience.
Advantages of this approach include shared language and scientific objectives within communities organized by subject, leaving more time for active collaboration on cutting-edge science.
On the other hand, homogeneity may lead to \textit{group think} and inhibit new, creative solutions.
In this case, it may be advantageous to design a hack week around a technique (e.g.\ Gaussian Processes) or modality (e.g.\ imaging), such as the ImageXD (image processing across domains\footnote{\url{http://http://www.imagexd.org/}}) meetings.
For these events, building a shared vocabulary and shared understanding of major data analysis problems is crucial, but they also allow for cross-disciplinary diffusion of techniques into other subjects and therefore decrease the risk of duplication of method development efforts.
\section*{Design considerations}

Design considerations depend very strongly on goals.
In general, longer events allow for a larger taught component, more ambitious projects and especially for cross-disciplinary events are more likely to provide enough time for different groups to effectively communicate across barriers of professional terminology.
On the other hand, because of their large participatory component, hack weeks tend to be exhausting, and long events may lead to fatigue among attendees, where active participation and measurable outcomes may drop sharply in the latter days of the workshops.
%This is particularly important if---as is often the case---participants continue participating in workshop-relevant activities even after the official component ends for the day.
Ways to mitigate these effects include capping taught components during the day, providing a reasonable clear schedule, and limiting parallel components to avoid decision fatigue.
%At Astro Hack Week, we have limited the scheduled taught components to no more than three out of nine hours for each of the five workshop days, and unscheduled (shorter) tutorials to a maximum of two per day.
%Even so, in our surveys, participants occasionally report concerns about choosing to attend tutorials versus working on projects.

Aside from time, space plays a particularly important role in facilitating a successful hack week.
Universities in general provide a convenient venue with existing structures to facilitate hack weeks (e.g. access to scientific publications, institutional support including staff and funding).
On the other hand, most spaces in universities are designed for lectures, which are diametrically opposed to unconference-style events.
Thus, finding an appropriate physical space within a university may be a challenge.
However, with the rise of active learning as a preferred teaching methods, traditional lecture spaces are transforming into more flexible spaces that are generally appropriate for hack weeks.
As a general rule, the smaller a hack week's emphasis on a taught component, the more flexible the space has to be, with ample opportunity for re-configuration.
%At Astro Hack Week, we have found it beneficial to spend one day mid-week at a different location (e.g. a company headquarter) to engage participants and break the routine.
%It is worth noting that collaborative spaces for non-traditional workshops are becoming more prevalent in academia (some attached to universities, some not) and may provide support and infrastructure that a traditional university location may not.

Another important design consideration is group size.
%With all previous hack weeks severely oversubscribed, it seems natural to simply admit more participants.
%However, this can counter the ideals of the workshop, in particular when building a community is one of its stated goals.
If the group is too large, participants are unlikely to even meet each other, and workshop cohesion may be lost as the workshop fractures into smaller groups, often among participants who already know each other.
This may inhibit knowledge transfer by clustering participants into small in-groups.
%Additionally, it is likely that the number participant-led components in the schedule may increase with workshop size.
%While generally desirable, a programme that is too crowded may lead to fatigue
On the other hand, if the size of the workshop is too small, it is unlikely to achieve the desired diversity among participants to foster new collaborations across sub-fields and disciplines.
In the past, we have found groups with sizes between 50 and 70 participants to be large enough to encourage a breadth of projects while allowing the workshop to function as a cohesive group.

As mentioned above, the balance between pedagogy and working depends both on the goals of the workshop and the topics around which the workshop is organized.
If participants have little shared knowledge, more teaching may be necessary in order to allow participants to effectively communicate with each other.
In communities where a shared understanding exists, tutorials can focus on more advanced or innovative topics, and less time may be allocated for them, leaving more time for active participation.
%In astronomy, a relatively small field, even students generally share a common knowledge base and have rudimentary knowledge about the types of data used and the challenges associated with each.
%Thus, we have focused the taught component of Astro Hack Week less on domain-specific knowledge, and instead offered tutorials in topics from domain-adjacent fields like statistics and computer science that attendees are unlikely to have encountered in their regular education.

Hack week outcomes, in turn, depend strongly on participants and are often a function of their interests and seniority.
some attendees arrive with the stated goal of writing a specific scientific article, usually more advanced participants with significant pre-knowledge of both hackathons and their topic of interest.
Many attendees arrive with the plan to learn a specific topic (such as machine learning) or bring a specific data set they believe the new knowledge may be applicable to.
This leads to a wide variety of project types from sandbox-style explorations to focused work efforts.
%It is worth noting that while a scientific paper need not be the stated goal of a hack (and is unlikely to be completed in the short time allocated in any case), results may still be published as short reports or unconference proceedings.
%For example, Neurohack week provides a venue for participants to publish a short (two-page) ''project report`` summarizing the hack that participants did during the week of NHW.
%Similarly, Python in Astronomy gathers all documents produced during the workshop (unconference transcripts, talk summaries, descriptions of sprint and hack projects) into citeable unconference proceedings.

\subsection*{Box 1: Impostor Syndrome}

The \textit{impostor phenomenon}, or \textit{impostor syndrome} (IS) is a dissonant feeling experienced by certain high-achieving individuals, that despite objective evidence to the contrary, they are in fact not as intelligent or capable as they appear.
Individuals with IS thus experience a fear of being ''found out``, shamed and expelled from their environment \cite{Clance1978-ef}.
Initial observations of the first Astro Hack Week conducted by data science ethnographer Brittany Fiore-Gartland\footnote{\url{http://astrohack week.org/blog/ethnographic-notes.html}} suggested that hack weeks are an environment prone to a particular kind of IS: participants might feel the need to be experts in multiple aspects of the activities pursued during a hack week: expertise in a scientific domain, as well as expertise in a the variety of technical tools used.
This particular form of IS hinges to some degree on the design focus on diversity of backgrounds (everyone else seems to know something that you do not!) and might be further exacerbated by the expectation that attendants expose their ideas to public scrutiny, find collaborators in a very short amount of time, and not least produce and present a successful hack at the end of the week\footnote{see also this insightful blog post: \url{https://medium.com/astronomy-without-stars/the-horror-of-hack-days-52c6b52cfc3b}}.
The prevalence of IS at a hack week may be endemic to the format, and should thus be a major concern for any organizing committee.
This is because of the chilling effect it tends to have on participants and the community as a whole, and particularly on women (in particular in fields in which women are under-represented) and members of ethnic and racial minorities, correlating with anxiety and other forms of mental distress \cite{Parkman2016-ro}.
Less severely, another major concern is that IS inhibits risk taking: participants experiencing it will be less likely to ask a question, to put forward an idea for a hack, to be pro-active about forming new collaborations.
 Many of the goals of a hack week, including the successful completion of projects, lateral knowledge transfer, as well as community building are hampered.
We are working within our hack weeks to mitigate IS using various techniques.
With respect to minority participants, ensuring adequate representation can decrease feelings of otherness and may help reduce IS.
More generally, being open about the presence and prevalence of IS can help participants feel more at ease.%(e.g. one Astro Hack Week participant remarked in a survey, \textit{''I really appreciated the direct acknowledgement of impostor syndrome on the first day.
%I think it helped ease the feeling!!``}).
Additionally, role models are very effective at encouraging positive behaviour: asking participants with prior experience at hack weeks at all academic levels to ask questions, even when they might know the answer, can help foster an inclusive environment that rewards risk taking.

\section*{Results}

\begin{figure*}[h!]
%\begin{center}
%\begin{subfigure}[t]{\textwidth}
\centering
%\caption{}
\includegraphics[width=\textwidth]{fig/f2.eps}
%\end{subfigure}

\caption{{\bf Post-workshop surveys from three hack weeks}: participants in the 2016 astro-, geo- and neuro- hack weeks responded to questions assessing their experiences. For GHW and NHW, response rates were 100\%, with $N_{\mathrm{GHW}}= 83$ and $N_{\mathrm{NHW}} = 86$, respectively. At AHW, the combined response rate for both years was 76\%, or $N_{\mathrm{AHW}} = 72$ out of $94$ total attendees. We report here about results in three different domains: the development of technical skills (a -- e), collaboration and learning (f,g), and shifts in attitudes towards reproducibility and open science (h -- j)}
\label{fig:survey}
%\end{center}
\end{figure*}

Measuring the success of a hack week objectively is complicated by the variety of objectives that a hack week might have (see above).
Additionally, the participant-driven, open-ended format facilitates knowledge transfer and collaborations in sometimes surprising ways that escape traditional measures of success.

One key metric is the number of publications that result from hack week projects, but this is a fairly narrow definition of success, in line with standard academic performance indicators.
Assuming that participants work largely in the open during a hack week, and that most projects have a strong programming component another indicator of success is the activity of participants in terms of code written and committed to a public code repository.
Still, these measures ignore learning, community-building as well as networking outcomes, which can be assessed through post-workshop surveys.
%It is in principle possible to measure gains in knowledge, networks and productivity within the pool of acceptable candidates for both those that attended the hack week and those that did not.
%This would then provide a somewhat objective measure of the impact the hack week has had.
%In practice, this has not been done for any of the hack weeks conducted thus far.
Here, we have taken an approach that combines these metrics: we start with survey results, and anecdotally report about publications, new code and projects generated (see the following section).
%Open-ended questions allow participants to provide feedback about outcomes, problems and goals that organizers had not anticipated.

%If hack week organizers plan to conduct research that involves hack week participants (for example, using assessments and evaluations in a paper such as this one) it is important to obtain approval from the Institutional Review Board (IRB), or equivalent body that approves research with human participants at the institution hosting the hack week.
%Though this research would usually fall under the category of ''minimal risk``, it is still important to establish procedures to manage these data, and to obtain informed consent.
%In particular, in some cases hack week organizers may be interested in studying not only the participants in the hack week, but also the applicants who did not end up participating.
%It is important to establish procedures to conduct such studies and to obtain approval from an IRB.

Focusing on the outcomes of astro-, geo- and neuro- hack weeks (AHW, GHW, NHW, respectively) from 2016 and 2017, we find that most participants self-report successful learning outcomes (AHW: 76\%, GHW: 89\%, NHW: 79\% for responses ``somewhat agree'', ``agree'' and ``strongly agree''; Figure \ref{fig:survey}, (a)).
The overwhelming majority of respondents at the hack weeks ($>95\%$ for all three events) believed that they learned things that improve their day-to-day research, and that attendance has made them a better scientist (Figure \ref{fig:survey}, (b, c) ).
More specifically, we compared learning outcomes in data visualization (the only topic explicitly shared between all three hack weeks; Figure \ref{fig:survey}, (d, e). We asked participants to subjectively rate their knowledge before the hack week, and find skill levels to be broadly distributed. This is expected, given the goal of diversity during the selection stage. We find that most participants have positive learning outcomes at all three hack weeks for data visualization, but that outcomes vary strongly, as is expected, too, with a group that is this diverse. We also note that there might be systematic effects and biases due to the subjectivity of rating knowledge and learning outcomes.

The majority of participants felt that they built valuable connections to other researchers (Figure \ref{fig:survey}, (g)), especially at Neuro Hack Week, where more than 64\% of participants strongly agreed with this statement.
Because peer learning is a major mode of knowledge transfer at hack weeks, we asked participants whether they taught other participants.
We find that again a majority agrees with this statement to some degree (AHW: 79\%, GHW: 69\%, NHW: 75\%; Figure \ref{fig:survey}, (f)), though responses are not as unequivocal as they are in some of the other categories.
Similarly to learning outcomes, this is expected given the range of skill levels at the workshops. It is notable, however, that we find no correlation between the response to this question and career stage for any of the events surveyed here. Participants at all career stages report similar engagement in teaching. This suggests there is some evidence for our hypothesis that compared to e.g.\ a traditional summer school, a hack week is less hierarchical and encourages lateral knowledge transfer between participants at different career stages.  We similarly find no correlations with gender identity or race/ethnicity.

We find that the hack weeks have been largely successful at efforts to promote positive attitudes towards reproducibility and open science: at all three events, more than 85\% of all participants (AHW: 86\%, GHW: 94\%, NHW: 95\%; Figure \ref{fig:survey}, (h)) put code or data created at the hack week into a public repository, while a substantially smaller fraction of participants followed a regular practice of publishing code before the event (Figure \ref{fig:survey}, (i)).
The overall behaviour reflects how general conventions and attitudes differ in different fields.
Astronomy shows the largest degree of openness toward open science, whereas our results indicate that open science is still fairly uncommon in the geosciences, with neuroscience falling in between.
%This implies that hack weeks can have the highest impacts in field where a priori engagement in reproducibility efforts is low and significant progress can be made towards changing researchers' attitudes during a collaborative workshop.
Similar attitudes are reflected when asking whether the hack week has made participants more comfortable with open science (Figure \ref{fig:survey}, (j)): again, geoscience shows the large improvement with over 97\% agreeing with this statement to some degree, followed by neuroscience (95\%) and astronomy (72\%).
Overall, our results indicate that hack weeks are effective at addressing persisting doubts about making research open and reproducible. While the focus on open science is not necessarily a required component of a hack week, it aligns naturally with many of the goals and values commonly promoted at hack weeks like open-source software and data sharing. In some fields, especially where ethical issues around data sharing and privacy are relevant, this might not be a desired focus of the hack week, or might be replaced or augmented by a discussion of ethical considerations.

%From the very nature of the activities that we encourage in hack weeks, participants in these events produce digital records of their research online. This means that it will be fairly straightforward to evaluate the long-term impact of these activities on participants' productivity (e.g., through contributions to open-source software) in the future.
Because all three events are relatively recent, it is still early to evaluate long-term outcomes, as well as others including publications and collaborations resulting from these events.
There are, however, initial indicators that all hack weeks encouraged long-term engagement with new concepts or tools and that they directly resulted in a number of publications \cite{gullysantiago2015,faria2016,keshavan2017,leonard2017,jordan2017,peterson2017,hahn2017,pricewhelan2017}. For specific examples, see also below.

\subsection*{Examples of Hack Week Outcomes}
\label{sec:outcomes}
\subsubsection*{Example 1: Astro Hack Week}
In 2015, a small team used AHW to found a new software project called Stingray\footnote{https://github.com/StingraySoftware/stingray} with the goal of providing well-tested implementations of time series analysis algorithms often used in X-ray astronomy.
The start of this project was facilitated by the collaborative environment at Astro Hack Week, including expertise in how to start/run open-source projects and role models of successful projects. Astro Hack Week enabled participants to seed a new collaboration around a software project needed by the larger community.
Stingray has since matured into an enduring collaboration within the community with five active maintainers and four Google Summer of Code projects.
\subsubsection*{Example 2: Geo Hack Week}
In 2016, a GHW project team used Google Earth Engine to explore spatial patterns in climate, topography and population data with the goal of mapping the most suitable locations for renewable energy sites in the United States.
The team used machine learning algorithms in conjunction with the powerful hardware resources provided by Google Earth Engine\footnote{\url{http://georgerichardson.net/2017/04/10/searching-for-energy-in-a-random-forest/}}.
George Richardson, one of the project leads, now works for a renewable resource company in Seattle.
\subsubsection*{Example 3: Neuro Hack Week}
Motion of study participants inside of the MRI machine is a major concern in neuroimaging studies, particularly in studies of children or patients, as they are more likely to move.
During NHW 2016 one of the teams focused on a large and openly available data-set of MRI data from children\footnote{ABIDE: \url{http://preprocessed-connectomes-project.org/abide}}.
To test the effect of motion on the results, the team conducted an analysis in which both the number of experimental subjects included, as well as motion cut-off were varied.
%They tested both the split-half reliability of an analysis of brain connectivity, as well as an analysis that used machine learning to distinguish between brains of children with and without autism spectrum disorder.
The team (composed of four different researchers from four institutions) continued to work on this project remotely after the end of NHW, and eventually published a paper describing these results in the open access journal Research Ideas and Outcomes \cite{leonard2017}.


\section*{Conclusions}

The fast-paced changes of the computational and methodological landscape require traditional fields of science to rapidly adapt to new data analysis challenges.
Traditional modes of learning, including university curricula, are often too slow to incorporate new developments on short enough time scales to meet their acute need in scientific advancement.
To address this imbalance, new types of workshops, including unconferences, hackathons and bootcamps, have been developed in recent years in various scientific disciplines to exist alongside with and support the existing structure of academic conferences.
Here, we introduce one such concept, hack weeks, and detail the underlying philosophical ideas along with experiences from events held in three different fields

As introduced above, hack weeks serve multiple purposes, including dissemination of state-of-the-art technological advances through the scientific community, building collaborations between academic subdisciplines and fostering interdisciplinary research as well as  promoting open science and reproducibility.
Initial results from three events held in 2016 and 2017 in three different fields (astronomy, geosciences and neurosciences) indicate that hack weeks succeed at all of these objectives, but that the measure of success is field-specific in that it depends to some degree on how much the concepts hack weeks promote were already adopted within the community.
Hack weeks are still a very young concept, and estimating the long-term impact of these events within the scientific communities they serve will require follow-up over multiple years to asses their effect on collaboration networks, career outcomes for early-career academics and adoption of new methods.
We have shown, however, that hack weeks provide an easy-to-implement, fairly low-cost method to introduce new technologies and methods into scientific fields on much shorter time scales than traditional teaching efforts can.
While we focus here on hack weeks in scientific fields, the concept could be extended to other areas, and is more generally useful in any area (1) where useful tools can be learned in short tutorials, (2) where results and outcomes can be produced on the timescale of a few days, and (3) that would benefit from collaborative approaches that cross traditional boundaries. Such areas could include for example the social sciences, music and art.



% If your first paragraph (i.e. with the \dropcap) contains a list environment (quote, quotation, theorem, definition, enumerate, itemize...), the line after the list may have some extra indentation. If this is the case, add \parshape=0 to the end of the list environment.
%\dropcap{T}his PNAS journal template is provided to help you write your work in the correct journal format.  Instructions for use are provided below.

%Note: please start your introduction without including the word ``Introduction'' as a section heading (except for math articles in the Physical Sciences section); this heading is implied in the first paragraphs.

%\section*{Guide to using this template on Overleaf}

%Please note that whilst this template provides a preview of the typeset manuscript for submission, to help in this preparation, it will not necessarily be the final publication layout. For more detailed information please see the \href{http://www.pnas.org/site/authors/format.xhtml}{PNAS Information for Authors}.

%If you have a question while using this template on Overleaf, please use the help menu (``?'') on the top bar to search for \href{https://www.overleaf.com/help}{help and tutorials}. You can also \href{https://www.overleaf.com/contact}{contact the Overleaf support team} at any time with specific questions about your manuscript or feedback on the template.

%\subsection*{Author Affiliations}

%Include department, institution, and complete address, with the ZIP/postal code, for each author. Use lower case letters to match authors with institutions, as shown in the example. Authors with an ORCID ID may supply this information at submission.

%\subsection*{Submitting Manuscripts}

%All authors must submit their articles at \href{http://www.pnascentral.org/cgi-bin/main.plex}{PNAScentral}. If you are using Overleaf to write your article, you can use the ``Submit to PNAS'' option in the top bar of the editor window.

%\subsection*{Format}

%Many authors find it useful to organize their manuscripts with the following order of sections;  Title, Author Affiliation, Keywords, Abstract, Significance Statement, Results, Discussion, Materials and methods, Acknowledgments, and References. Other orders and headings are permitted.

%\subsection*{Manuscript Length}

%PNAS generally uses a two-column format averaging 67 characters, including spaces, per line. The maximum length of a Direct Submission research article is six pages and a PNAS PLUS research article is ten pages including all text, spaces, and the number of characters displaced by figures, tables, and equations.  When submitting tables, figures, and/or equations in addition to text, keep the text for your manuscript under 39,000 characters (including spaces) for Direct Submissions and 72,000 characters (including spaces) for PNAS PLUS.

%\subsection*{References}

%References should be cited in numerical order as they appear in text; this will be done automatically via bibtex, e.g. \cite{belkin2002using} and \cite{berard1994embedding,coifman2005geometric}. All references, including for the SI, should be included in the main manuscript file. References appearing in both sections should not be duplicated.  SI references included in tables should be included with the main reference section.

%\subsection*{Data Archival}

%PNAS must be able to archive the data essential to a published article. Where such archiving is not possible, deposition of data in public databases, such as GenBank, ArrayExpress, Protein Data Bank, Unidata, and others outlined in the Information for Authors, is acceptable.

%\subsection*{Language-Editing Services}
%Prior to submission, authors who believe their manuscripts would benefit from professional editing are encouraged to use a language-editing service (see list at www.pnas.org/site/authors/language-editing.xhtml). PNAS does not take responsibility for or endorse these services, and their use has no bearing on acceptance of a manuscript for publication.

%\begin{figure}%[tbhp]
%\centering
%\includegraphics[width=.8\linewidth]{frog}
%\caption{Placeholder image of a frog with a long example caption to show justification setting.}
%\label{fig:frog}
%\end{figure}

%\subsection*{Digital Figures}
%\label{sec:figures}

%Only TIFF, EPS, and high-resolution PDF for Mac or PC are allowed for figures that will appear in the main text, and images must be final size. Authors may submit U3D or PRC files for 3D images; these must be accompanied by 2D representations in TIFF, EPS, or high-resolution PDF format.  Color images must be in RGB (red, green, blue) mode. Include the font files for any text.

%Figures and Tables should be labelled and referenced in the standard way using the \verb|\label{}| and \verb|\ref{}| commands.

%Figure \ref{fig:frog} shows an example of how to insert a column-wide figure. To insert a figure wider than one column, please use the \verb|\begin{figure*}...\end{figure*}| environment. Figures wider than one column should be sized to 11.4 cm or 17.8 cm wide.

%\subsection*{Single column equations}

%Authors may use 1- or 2-column equations in their article, according to their preference.

%To allow an equation to span both columns, options are to use the \verb|\begin{figure*}...\end{figure*}| environment mentioned above for figures, or to use the \verb|\begin{widetext}...\end{widetext}| environment as shown in equation \ref{eqn:example} below.

%Please note that this option may run into problems with floats and footnotes, as mentioned in the \href{http://texdoc.net/pkg/cuted}{cuted package documentation}. In the case of problems with footnotes, it may be possible to correct the situation using commands \verb|\footnotemark| and \verb|\footnotetext|.

%% Do not use widetext if paper is in single column.
%\begin{widetext}
%\begin{align*}
%(x+y)^3&=(x+y)(x+y)^2\\
%       &=(x+y)(x^2+2xy+y^2) \numberthis \label{eqn:example} \\
%       &=x^3+3x^2y+3xy^3+x^3.
%\end{align*}
%\end{widetext}

%\begin{table}%[tbhp]
%\centering
%\caption{Comparison of the fitted potential energy surfaces and ab initio benchmark electronic energy calculations}
%\begin{tabular}{lrrr}
%Species & CBS & CV & G3 \\
%\midrule
%1. Acetaldehyde & 0.0 & 0.0 & 0.0 \\
%2. Vinyl alcohol & 9.1 & 9.6 & 13.5 \\
%3. Hydroxyethylidene & 50.8 & 51.2 & 54.0\\
%\bottomrule
%\end{tabular}

%\addtabletext{nomenclature for the TSs refers to the numbered species in the table.}
%\end{table}

%\subsection*{Supporting Information (SI)}

%The main text of the paper must stand on its own without the SI. Refer to SI in the manuscript at an appropriate point in the text. Number supporting figures and tables starting with S1, S2, etc. Authors are limited to no more than 10 SI files, not including movie files. Authors who place detailed materials and methods in SI must provide sufficient detail in the main text methods to enable a reader to follow the logic of the procedures and results and also must reference the online methods. If a paper is fundamentally a study of a new method or technique, then the methods must be described completely in the main text. Because PNAS edits SI and composes it into a single PDF, authors must provide the following file formats only.

%\subsubsection*{SI Text}

%Supply Word, RTF, or LaTeX files (LaTeX files must be accompanied by a PDF with the same file name for visual reference).

%\subsubsection*{SI Figures}

%Provide a brief legend for each supporting figure after the supporting text. Provide figure images in TIFF, EPS, high-resolution PDF, JPEG, or GIF format; figures may not be embedded in manuscript text. When saving TIFF files, use only LZW compression; do not use JPEG compression. Do not save figure numbers, legends, or author names as part of the image. Composite figures must be pre-assembled.

%\subsubsection*{3D Figures}

%Supply a composable U3D or PRC file so that it may be edited and composed. Authors may submit a PDF file but please note it will be published in raw format and will not be edited or composed.

%\subsubsection*{SI Tables}

%Supply Word, RTF, or LaTeX files (LaTeX files must be accompanied by a PDF with the same file name for visual reference); include only one table per file. Do not use tabs or spaces to separate columns in Word tables.

%\subsubsection*{SI Datasets}

%Supply Excel (.xls), RTF, or PDF files. This file type will be published in raw format and will not be edited or composed.

%\subsubsection*{SI Movies}

%Supply Audio Video Interleave (avi), Quicktime (mov), Windows Media (wmv), animated GIF (gif), or MPEG files and submit a brief legend for each movie in a Word or RTF file. All movies should be submitted at the desired reproduction size and length. Movies should be no more than 10 MB in size.

%\subsubsection*{Still images}

%Authors must provide a still image from each video file. Supply TIFF, EPS, high-resolution PDF, JPEG, or GIF files.

%\subsubsection*{Appendices}

%PNAS prefers that authors submit individual source files to ensure readability. If this is not possible, supply a single PDF file that contains all of the SI associated with the paper. This file type will be published in raw format and will not be edited or composed.

\matmethods{
We performed post-attendance surveys for AHW, GHW and NHW in 2016 and 2017. All surveys contained general questions about attitudes towards the workshop as well as open science and reproducibility, shared among all three surveys. Response rates for NHW (2016: 41 responses; 2017: 45 responses) and GHW (2016: 42 responses; 2017: 41 responses) were 100\% in both years; the response rate for AHW was 71\% (35 out of 49) in 2016 and 82\% (37 out of 45) in 2017. Participants were asked to respond to statements regarding these topics using a six-point Likert-type scale. All questions were anonymously recorded.  The experimental procedures were approved by the Institutional Review Boards at UW, NYU and UC Berkeley. All participants gave their informed consent.
No responses were discarded, and no pre-processing was performed on the data. We test for correlations between demographic characteristics (independent variable) and question responses (dependent variable) using a standard $\chi^2$ test and compute the effect sizes via a bias-corrected version of Cram\'{e}r's V \citep{cramer1946,bergsma2013}, denoted $\phi_c$. We additionally perform equivalence tests on the effect size to quantify the absence of correlations. The full procedure is available in the SM (Section 1) and online\footnote{See the repository: \url{https://github.com/uwescience/HackWeek-Writeup}}.

%We performed post-attendance surveys for AHW, GHW and NHW in 2016 and 2017. All surveys contained general questions about attitudes towards the workshop as well as open science and reproducibility, and their own skills in statistical and computational methods, shared among all three surveys. The NHW and GHW surveys and the AHW 2017 survey were administered on site on the last day of the workshop; AHW participants in 2016 were e-mailed immediately after the workshop, with reminders after several weeks. All responses were collected within four weeks of the end of the workshop. The experimental procedures were approved by the Institutional Review Boards at UW, NYU and UC Berkeley. All participants gave their informed consent.

%Response rates for NHW (2016: 41 responses; 2017: 45 responses) and GHW (2016: 42 responses; 2017: 41 responses) were 100\% in both years; the response rate for AHW was 71\% (35 out of 49) in 2016 and 82\% (37 out of 45) in 2017. The lower response rates for AHW can be explained by the generally lower rates of response for time-delayed survey in 2016. In 2017, a number of participants did not attend the full week, and thus several attendees were not present on the last day when the survey was administered. For the AHW surveys, we checked whether the demographics of the respondents different significantly from those of the attendees overall in three relevant categories (career stage, racial/ethnic background and gender identity) and find no significant difference for any of the groups except for career stage in 2017: here, we find that graduate students are underrepresented (27\% of respondents compared to 41\% of attendees), while undergraduate students are overrepresented (25\% of respondents compared to 12\% of attendees). Because we group responses in the career stage category into categories ``early-career'' (students and postdoctoral fellows), ``late-career'' (faculty, research scientists) and ``other'' (e.g. participants from outside academia), this discrepancy has no bearing on the results we report. For gender, we grouped participants into ``minority'' (women and non-binary) and ``non-minority'' (men) participants; similarly, for race and ethnicity, we also chose ``non-white'' (all but Caucasian participants) and ``white'' (Cauciasian) categories. Note that demographic data for GHW and NHW is only available for 2016, for AHW for both 2016 and 2017, and responding to any question was optional.

%Participants were asked to respond to statements regarding these topics using a six-point Likert-type scale. All questions were anonymously recorded. No responses were discarded, and no pre-processing was performed on the data. We test for correlations between demographic characteristics (independent variable) and question responses (dependent variable) using a standard $\chi^2$ test and compute the effect sizes via a bias-corrected version of Cram\'{e}r's V \citep{cramer1946,bergsma2013}, denoted $\phi_c$ in the manuscript. For a fairly permissive significant threshold of $\alpha < 0.05$, we require a $p$-value of $p < 0.0007$, corrected for $N_\mathrm{trials} = 36$ trials, to claim a significant effect as well as at least a small effect size according to~\citep{cohen1988}. The full analysis procedure is available online\footnote{See the repository: \url{https://github.com/uwescience/HackWeek-Writeup}}.

%\subsection*{Subsection for Method}
%Example text for subsection.
}

\showmatmethods % Display the Materials and Methods section

\acknow{The authors would like to thank the anonymous reviewers and the editor for their helpful comments and suggestions, Laura Nor\'{e}n for help on ethics and IRB, Stuart Geiger for helping to formulate the survey, Christine Huebner for advice on statistics, Brittany Fiore-Gartland, Laura Nor\'{e}n and Jason Yeatman for comments on the manuscript, and Tal Yarkoni for advice regarding automated selection procedures. This work was partially supported by the Moore-Sloan Data Science Environments at UC Berkeley, New York University, and the University of Washington. NHW is supported through a grant from the National Institute for Mental Health (1R25MH112480). DH is partially supported by the James Arthur Postdoctoral Fellowship at NYU, and acknowledges support from the DIRAC Institute in the Department of Astronomy at the University of Washington. The DIRAC Institute is supported through generous gifts from the Charles and Lisa Simonyi Fund for Arts and Sciences, and the Washington Research Foundation.}

\showacknow % Display the acknowledgments section

% \pnasbreak splits and balances the columns before the references.
% If you see unexpected formatting errors, try commenting out this line
% as it can run into problems with floats and footnotes on the final page.
\vspace{-1cm}
\pnasbreak

% Bibliography
\bibliography{paper}

\end{document}
