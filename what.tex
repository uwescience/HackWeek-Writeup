\section*{What is a hack week?}

Our hack weeks combine structured, tutorial-style instruction with informal education and peer learning opportunities occurring within projects and hacks. Additionally, they provide opportunities to build new collaborations and network. In a space spanned by pedagogical focus on one axis and focus on project work on the other, the hack weeks we have organized are designed to lie somewhere in between traditional summer schools and hackathons, where we believe they fill a space not currently fully addressed by existing models (Figure \ref{fig:hackspectrum}).

The hackathon, a time-bounded, collaborative events that bring together participants around a shared challenge or learning objective \cite{Decker2015}, forms one primary axis of our events. 
Hackathons originated from the open-source software movement and have historically focused on software and technology development. 
In recent years hackathons have evolved into a model providing opportunities for intensive, interdisciplinary collaboration \cite{Groen2015-cj} and education \cite{Kienzler2015-zu,Lamers2014-xf} in the sciences. 
Core element of all hackathons include opportunities for networking, strengthening of social ties, and the building of community connections, both within and across disciplines.
Building on these core elements hackathons have been implemented in different ways depending on the overall purpose, mode of participation, style of work environment and participant motivation \cite{Drouhard2017}. 
%``Catalytic'' hackathons seek novel project ideas aimed at solving a tractable, well-defined challenge.
%``Contributive'' hackathons seek to improve to an existing effort through focused work on discrete tasks, for example to make up for deficiencies in an ongoing project.
%Finally, ``Communal'' hackathons place a strong focus on building a culture of practice and developing resources within an existing community, often defined by a specific domain of knowledge.

Summer schools have been designed to excel in transfer of knowledge from experts in the field to (early-career) researchers: they often serve as an entry point for scientists who aim to expand their research into a new area or switch fields. They are excellent at giving participants a reasonably deep understanding of a topic or field in a short amount of time. Within this concept, learning can take many forms, including traditional lecture formats but also hands-on project work, often in teams (e.g., Advanced Course on Computational Neuroscience, Okinawa Computational Neuroscience Course, Woods Hole Computational Neuroscience Summer course). 

Our hack weeks extend the scientifically-focused, communal hackathon model into a space that includes a strong element of pedagogy and peer learning. They aim to synthesize different goals and strategies from both models: they are much more participant-driven than a summer school, but have a stronger focus on pedagogy than a hackathon. Where a summer school is often organized around a framework of lectures and tutorials known in advance, hack weeks leave the majority of time unstructured. Tutorials at hack weeks often serve as an entry point into a topic for further exploration and learning, rather than an in-depth study. This is closely related to the nature of the material taught at these events. Data science is a new, emerging area combining knowledge and skills from many different fields, only a fraction of which can be communicated within a short tutorial. As organizers, we make a curated selection of a small number of tutorials each year, and encourage participants to use the unstructured time to self-organize tutorials on any topic as needed.%, often on topics vastly different to those discussed in the formal lectures. 

The unstructured schedule and inherent lack of focus on specific topics within a hack week present both an opportunity and a risk. They uniquely allow organizers to tailor the content of the workshop to the needs of the participants in an ad-hoc fashion, including the number of tutorials and their content. This way, the group as a whole can respond quickly or react to unforeseen challenges and opportunities. They encourage participants to self-organize in many different forms: experts working with other experts, mentoring relationships between experts and non-experts, or study groups among non-experts, to name but a few. Hack weeks also allow participants to experiment with projects and ideas beyond their day-to-day research: our hack weeks explicitly encourage projects e.g.~ around outreach and work aimed at improving the scientific community itself. 

There is, however, a major risk in the lack of focus: by wanting to be too many things at once, a hack week can run the danger of not doing any of them well. Because tutorials are not the major focus of a hack week, the knowledge gained by participants in these tutorials may not be very deep. Additionally, a hack week carries a much larger risk of failure if objectives and expectations are not set by organizers well in advance, and clearly communicated to participants, because they often require more preparation from the side of participants than other events (e.g. define project ideas, research possible connections to other participants). 
Because of these risks, organizers face a much larger degree of uncertainty both before and during a hack week, and need to be prepared to focus much of their energy on thoughtful selection and management of participants, keeping track of the goals different attendees may have, and careful facilitation of the large range of different types of activities at any given time (see also the following sections as well as the SM, Section 4).

We note that the terminology for these events is constantly evolving, and that the ``hackathon'' concept may have implicit connotations that are disfavored in some communities. We also note that all of these events live on a constantly changing continuum, depending on the requirements of the scientific domain within which they live. For example, Neuro Hack Week is moving toward a more traditional summer school model, while Astro Hack Week has strengthened its focus on projects and hacks in recent iterations.
