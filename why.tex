\section*{Why run a Hack Week?}

\noindent\textbf{Education and Training}: While some hack weeks are more focused more on education than others (see Figure \ref{fig:hackspectrum}), skill-development in the form of tutorials as well as informal and peer learning is often a component. Furthermore, lateral knowledge transfer \cite{b:wilson-swc-lessons-2016} through collaboration provides an opportunity to learn skills that are not described in papers and software implementations.

\noindent\textbf{Tool Development}: Hack weeks present an opportunity for scientific software developers to meaningfully engage with users and critically evaluate applications to particular scientific issues.

\noindent\textbf{Community Building}: Hack weeks are an opportunity to catalyze community development through a shared interest in solving computational challenges with open source software. They allow computationally-minded researchers to break from the isolation of their institutions and spark new collaborations.

\noindent\textbf{Interdisciplinary research}: Intensive, time-bounded collaborative events are an opportunity to experiment with concepts, questions, and methods that span boundaries within and across disciplines. Despite the fact that interdisciplinary experiments are impactful \cite{Hall2012-hi}, they are often discouraged in traditional academic environments \cite{Sung2003-go}.

\noindent\textbf{Recruitment and Networking}: Hack weeks are a melting pot of participants from academia, government, and industry and provide numerous opportunities for networking. Close collaboration in diverse groups exposes skills that might be suitable for careers outside of a narrow domain.

\noindent\textbf{It's fun}: Hack weeks provide a respite from routine and a low-stress venue to learn new skills and attempt high-risk projects.

\noindent Note that the reasons for participants to attend a hack week are as diverse as the reasons for running such an event.
Beginner participants may attend primarily to learn a new technique, while others may attend to gain experience in mentoring, or to focus on an existing project already in progress (for more details on setting objectives, see SM, Section 4.1.2).
