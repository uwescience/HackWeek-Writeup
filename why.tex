\section*{Why run a Hack Week?}

There are several reasons to run a hack week of the sort described here.

\begin{itemize}
\item{\textit{Education and Training}:
While some hack weeks are focused more on education than others (see Figure \ref{fig:hackspectrum}), there is often a skill-development component that entails discussions about reproducible research. Participants gain a strong foundation in open science practices from the diverse group setting and go on the become ambassadors for such practices. This type of lateral knowledge transfer is a core attribute of a hack week, and provides an opportunity to learn skills that are not described in papers and software implementations.}

\item{\textit{Tool Development}: Hack weeks present an opportunity for scientific software developers to meaningfully engage with users and critically evaluate applications to particular scientific issues.}

\item{\textit{Community Building}: Hack weeks provide a tremendous opportunity to catalyze community development through a shared interest in solving computational challenges with open source software. These events allow computationally minded researchers to break from the isolation of their academic departments and spark new collaborations.}


\item{\textit{Interdisciplinary research}: Intensive, time-bounded collaborative events are an excellent opportunity to experiment with concepts, questions, and methods that span boundaries within and across disciplines. Despite the fact that interdisciplinary experiments are impactful \cite{Hall2012-hi}, they are often discouraged in risk averse traditional academia \cite{Sung2003-go}}.

\item{\textit{Recruitment and Networking}: Hack weeks are often a melting pot of participants from academia, government, and industry and provide numerous opportunities for networking. Close collaboration in diverse groups exposes skills that might be suitable for careers outside of one's narrow domain.}

\item{\textit{It's fun}: Hack weeks provide a respite from day-to-day research activities and provide a low-stress venue to learn new skills and attempt high-risk projects.}

\end{itemize}

\noindent It is worth noting that the reasons for participants to attend a hack week are as diverse as the reasons for running such an event.
Beginner participants may attend primarily to learn a new technique, while others may attend to gain experience in mentoring, or to focusing on an existing project already in progress (for more details on setting objectives, see supplementary materials).
This leads to a wide variety of project types from sandbox-style explorations to focused work efforts.
