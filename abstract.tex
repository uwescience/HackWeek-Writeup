\begin{abstract}
Across many scientific disciplines, methods for recording, storing and analyzing data are rapidly increasing in complexity.
Facility in data science tools to manage this complexity requires training in new programming languages and frameworks, as well as immersion in new modes of interaction that foster data sharing and collaborative software development and exchange across disciplines.
Learning these skills from traditional university curricula can be challenging because most courses are not designed to evolve on time scales that can keep pace with rapidly shifting data science methods.
Here we present the concept of a hack week as a novel and effective model offering opportunitites for networking and community building, education in state-of-the-art data science methods and immersion in collaborative project work.
We find that hack weeks are successful at cultivating collaboration and facilitating the exchange of knowledge.
Participants self-report that these events help them both in their day-to-day research as well as their careers.
Based on our results, we conclude that hack weeks present an effective, easy-to-implement, fairly low-cost tool to positively impact data analysis literacy in academic disciplines, foster collaboration and cultivate best practices.
\end{abstract}
